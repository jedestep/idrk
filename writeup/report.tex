\documentclass{article}
\begin{document}
\title{600.639 Computational Genomics\\
Final Project}
\date{}
\author{Ashleigh Thomas and Jed Estep}
\maketitle

\begin{abstract}
Easily searchable representations of genome strings are useful for many kinds of analysis, but in practice their usability is often limited on commodity hardware due to their high memory requirements. Suffix arrays are one of the least memory-intensive commonly used representations, but its space requirements may still be prohibitive in the case of indexing numerous genomes. In this paper we investigate the suffix array compression scheme described in \cite{GV05}. We attempt to apply the compressed suffix arrays as a searchable database of multiple genomes with use in the context of metagenomics. The design of our database is similar to that of QUASAR \cite{B99}.
\end{abstract}
\section{Introduction}
We arrived at this design while investigating multiple topics. From one end, we were interested in pursuing the applicability of compressed data structures to representing genomes. Many implementations of useful index structures like suffix trees are extremely memory intensive, so decreasing the size of their representation is paramount if they are to be used on commodity hardware. \cite{B99} points out that, in the operation of QUASAR, numerous special methods are necessary to accommodate suffix arrays which are too large to fit in main memory. As such, we attempted to apply the compression methods of \cite{GV05} to a search index similar to QUASAR.\\


\begin{thebibliography}{1}
\bibitem{GV05}
	Roberto Grossi and Jeffrey Vitter,
	\emph{Compressed Suffix Arrays and Suffix Trees with Applications to Text Indexing and String Matching}.
	Society for Industrial and Applied Mathematics Journal of Computing,
	Vol. 35, No. 2, pp. 378-407,
	2005.
\bibitem{B99}
	Stefan Burkhardt, et al.,
	\emph{q-gram Based Database Searching Using a Suffix Array (QUASAR)}.
	Proceedings of the third annual international conference on Computational molecular biology,
	pp. 77-83,
	1999.
\end{thebibliography}
\end{document}